\chapter{Objectifs} \label{ch:objectifs}

Nous cherchons à produire un modèle numérique cohérent qui fournirait des résultats que l'on a pu observer par le passé. Le modèle construit son propre monde mais en se basant sur une certaine réalité. Par exemple, la notion de génome implémentée dans ce modèle n'est pas représentative de la réalité. En effet le génome d'un individu n'est pas aussi simple qu'un seul entier. Il s'agit donc ici de créer une abstraction simplifiée de fonctionnements réels et de l'implanter dans le modèle afin de simuler des processus réels très complexes. Cette simplification permet une analyse plus aisée des résultats. En effet, les corrélations et causalités sont plus facilement interprétables lorque le modèle est simplifié.\\

La validité du modèle peut aussi se tester en le soumettant à des cas simples dont on peut inférer les résultats. Connaissant à l'avance l'orientation de la simulation, nous pouvons vérifier que le programme fournisse bien des résultats conformément aux attentes. Pour ce faire on peut se référer aux modèles compartimentaux en épidémiologie pour tester notre modèle numérique. A quel point le modèle reflète la réalité que nous pourrions observer dans les modèles compartimentaux ? La validité du modèle sur des cas connus est nécessaire pour conclure de nouveaux résultats. Les modèles mathématiques compartimentaux permettent aussi d'explorer les mécaniques du modèle et de mesurer ses comportements et phénomènes.\\

Le but d'une simulation est de fournir des résultats. Quels résultats pouvons-nous extraire d'une simulation ? Nous pouvons observer des différences dans les résultats en faisant varier les paramètres du système. Étant donné que le modèle possède sa propre réalité, nous ne pouvons extraire des informations qu'en comparant des résultats d'une simulation avec une autre. Il s'agit donc de mesurer et quantifier des variations de comportements en faisant varier un ou plusieurs paramètres du modèle. La recherche se concentre sur la notion de diversité, introduite dans le modèle par les génomes des acteurs. Le facteur de diversité est le paramètre principal étudié. Quel est l'impact de la diversité d'une population sur la propagation d'une épidémie ? La variation d'autres paramètres peuvent aussi être étudié comme par exemple la vitesse de mutation des agents pathogènes. Il s'agit donc d'illustrer et de quantifier l'impact qu'ont ces paramètres sur l'émergence d'événement de grande taille comme les pandémies.\\

Un modèle performant est nécessaire pour effectuer des simulations de grande taille. Dépendant des performances du modèle, quelles sont les ordres de grandeur des simulations effectuées ? et quelle quantité d'expériences pouvons-nous effectuer pour construire des statistiques ? En effet le travail cherche à observer des événement à grande échelle que le modèle doit supporter et pouvoir répéter un certain nombre de fois.\\

L'objectif du travail est de fournir un modèle numérique capable de faire des prédictions de propagation de pandémies dans une population donnée en se basant sur la notion de diversité. Pour quel type de problème pourrons nous faire de la prédiction ? Le modèle ne se limite pas à deux espèces en particulier mais à deux catégories d'espèces. Nous pouvons donc appliquer le modèle à un grand nombre de cas.\\

Le dernier objectif de la recherche est de pouvoir faire de la prédiction explicative. Contrairement à la prédiction anticipative, notre modèle pourrait permettre d'expliquer l'émergence ou non d'événement à grande échelle en fonction de différents paramètres. Il s'agirait de pouvoir expliquer les facteurs qui ont permis la propagation d'un pathogène.