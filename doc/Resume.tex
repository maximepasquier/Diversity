\chapter{Résumé} \label{ch:resume}

Une pandémie est une épidémie présente sur une grande zone et affectant un grand nombre de personnes. Ces épidémies sont le résultat de plusieurs causes. L'espèce humaine est touchée par des pandémies à une fréquence d'une fois par siècle mais depuis les années 2000, ce rythme s'est grandement accéléré. En effet, nous avons été victime de pandémies comme Ebola en 2013, SARS-2 en 2008, COVID en 2020 et bien d'autres. Ces épidémies à grande échelle deviennent de plus en plus fréquentes pour une multitude de raisons.\\

Le travail se concentre sur l'élaboration d'un modèle numérique de propagation de pandémie. L'objectif est de construire ce modèle en incorporant la notion de diversité écologique. La diversité d'une espèce est un facteur clef permettant sa survie et son évolution. Par conséquent nous voulions quantifier l'impact de la diversité sur la propagation de pandémies. Le premier objectif est de valider le modèle en le confrontant à des modèles compartimentaux déjà prouvés. Il s'agit de soumettre le modèle à différents examens, de comprendre son fonctionnent et de pouvoir l'expliquer. Le modèle contient une multitude de paramètre, le travail n'en couvre que quelques-uns.\\

Les simulations se basent sur le principe d'automate cellulaire. Le système se présente sous forme d'une grille bidimensionnelle tel un échiquier de taille voulu. La population navigue sur cet échiquier et les individus interagissent les uns avec les autres. Le but est de simuler les mouvements et les contacts au sein d'une population qui permettraient la propagation d'agents pathogènes. Chaque simulation est paramétrable et les données sont enregistrées pour être analysées.\\

L'objectif est d'étudier les résultats de simulations en faisant varier les paramètres du modèle dont la diversité écologique. Les résultats permettraient de quantifier l'impact qu'à la diversité écologique sur une pandémie.