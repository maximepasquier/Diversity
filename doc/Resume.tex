\chapter{Résumé} \label{ch:resume}

Une pandémie est une épidémie présente sur une grande zone et affectant un grand nombre de personnes. Ces épidémies sont le résultat de plusieurs causes. L'espèce humaine est touchée par des pandémies à une fréquence d'une fois par siècle mais depuis les années 2000, ce rythme s'est grandement accéléré. En effet, nous avons été victime de pandémies comme Ebola en 2013, SARS-2 en 2008, COVID en 2020 et bien d'autres. Ces épidémies à grande échelle deviennent de plus en plus fréquentes pour une multitude de raisons.\\

Le travail se concentre sur l'élaboration d'un modèle numérique de propagation de pandémie. L'objectif est de construire ce modèle en incorporant la notion de diversité écologique. La diversité d'une espèce est un facteur clef permettant sa survie et son évolution. Par conséquent, le travail explore les différentes influences de la diversité sur les systèmes.\\

Les simulations se basent sur le principe d'automate cellulaire. Le système se présente sous forme d'une grille bidimensionnelle tel un échiquier de taille définie. La population navigue sur cet échiquier et les individus interagissent les uns avec les autres. Le but est de simuler les mouvements et les contacts au sein d'une population qui permettraient la propagation d'agents pathogènes. Chaque simulation est paramétrable et les données sont enregistrées pour être analysées.\\

La recherche est structurée en trois parties. La première partie consiste à valider l'implémentation à l'aide de modèles épidémiologiques déjà existants. Il s'agit de valider le modèle en le confrontant à des modèles compartimentaux déjà prouvés. La seconde partie explore les mécaniques du modèles et ses fonctionnements. Il s'agit de soumettre le modèle à différents examens, de comprendre son fonctionnement et de pouvoir l'expliquer. Le modèle contient une multitude de paramètres, le travail n'en couvre que quelques-uns. Finalement la troisième partie mesure l'influence des mécanismes de diversité et mutation sur les pandémies.\\

Les résultats montrent que le modèle est capable de s'adapter aux modèles épidémiologiques et d'en imiter les comportements. De plus, l'exploration des mécanismes du modèle et leur explication et illustration permet de comprendre son fonctionnement. Finalement les mesures de diversité et mutation suivent nos intuitions. La diversité est un atout s'opposant à la propagation de pandémies et au contraire la mutation est une arme permettant de s'adapter à son environnement. Les résultats des simulations montrent des variations importantes lorsque ces mécanismes sont activés.