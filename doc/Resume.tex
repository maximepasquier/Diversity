\chapter{Résumé} \label{ch:resume}

Une pandémie est une épidémie présente sur une grande zone et affectant un grand nombre de personnes. Ces épidémies peuvent être causée par de multiples facteurs. L'espèce humaine est touchée par des pandémies à une fréquence d'une fois par siècle mais depuis les années 2000, ce rythme c'est grandement accéléré. En effet, nous avons été victime de pandémies comme Ebola en 2013, SARS-2 en 2008, COVID en 2020 et bien d'autres. Ces épidémies à grande échelle deviennent de plus en plus fréquentes pour plusieurs raisons que nous ne traiterons par ici.\\

Le travail se concentre sur l'aspect de diversité écologique des populations touchées lors de pandémies. La diversité d'une espèce est un facteur clef permettant sa survie et son évolution. L'objectif de l'étude est de constater l'impacte de la diversité des populations sur la propagations de pandémies. En effet, une grande diversité au sein d'une population ralenti la propagation d'une pandémie. Il s'agit dans ce travail d'étudier les différentes issues de simulations en modifiant le paramètre de diversité de la population étudiée.\\

L'étude se concentre sur la réalisation d'une simulation permettant de modéliser une population avec des caractéristiques paramétrables. Nous contaminons un individu et observons au fil du temps la propagation ou non de l'épidémie.\\

La simulation se base sur le principe d'automate cellulaire. Le système se présente sous forme d'une grille bidimensionnelle tel un échiquier de taille voulu. La population navigue sur cet échiquier et les individus interagissent les uns avec les autres. Le but est de simuler les mouvements au sein d'une population et les contactes permettant la propagation d'agents pathogènes. Au fil du temps de la simulation nous pouvons étudier la propagation de la pandémie en fonction des paramètres du modèle.\\

L'objectif est d'étudier les résultats de simulations en faisant varier la diversité de la population ainsi que d'autres facteurs. Les résultats permettraient de quantifier l'impacte qu'à la diversité d'une espèce sur une éventuelle pandémie.
