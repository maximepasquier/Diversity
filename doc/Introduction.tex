\chapter{Introduction} \label{ch:introduction}

L'élément principal de la recherche scientifique est de valider un modèle numérique, de comprendre son fonctionnement, et d'étudier la propagation de pandémies en fonction de différents facteurs dont la diversité. Pour des questions de simplification, beaucoup de facteurs externes ont été ignorés. En effet, l'apparition et la propagation d'une épidémie sont le résultat d'une multitude de facteurs complexes. Les simulations se basent exclusivement sur des paramètres simples tout en ignorant des aspects comme l'âge, la condition de vie ou encore l'accès aux soins. Le modèle cherche à simuler des mécanismes comme les mutations et les immunités. D'autres facteurs sont paramétrables, comme la taille du système, le nombre d'individus et bien d'autres. En un premier temps le travail se concentre sur la validation et l'exploration du modèle, puis cherche à démontrer l'impact qu'ont les mécanismes de diversité sur l'apparition d'événements de grande ampleur.\\

La simulation ne comporte que deux types d'acteurs : des individus d'une population donnée et des agents pathogènes. La dimension spatiale est représentée par une grille bidimensionnelle permettant aux individus de se déplacer librement. Ce modèle simplifié permet de modéliser les interactions entre les acteurs du système.\\

Initialement, un seul individu du système est contaminé, il s'agit du patient zéro. La simulation fait évoluer le système pendant un certain temps et nous cherchons à observer si l'agent pathogène parvient à se propager ou non parmi la population d'individu. Une multitude d'autres paramètres peuvent être modifiés pour influencer la simulation.\\

La recherche principale est d'observer et de mesurer l'évolution du système en faisant varier les diversités des acteurs, en sachant que chaque acteur du système possède un indicateur représentant son code génétique.\\

La question que l'on se pose pour ce travail est de déterminer si la diversité d'une population touchée par une épidémie modifie la propagation d'un agent pathogène ou non. Il s'agit ici d'illustrer et de quantifier cet effet à l'aide d'un modèle numérique.