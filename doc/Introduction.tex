\chapter{Introduction} \label{ch:introduction}

L'élément principal de la recherche scientifique est l'étude de propagation de pandémies en fonction de la diversité de la population étudiée. Pour des questions de simplification, beaucoup de facteurs externes ont été ignorés. En effet, l'apparition et la propagation d'une épidémie est le résultat d'une multitude de facteurs complexes. La simulation se base exclusivement sur des paramètres simples tout en ignorant des aspects comme l'âge, la condition de vie ou encore l'accès aux soins. Le modèle se focalise sur le principe de génomes, mutations et immunités. D'autres facteurs sont paramétrables, comme la taille du système ou le nombre de personnes et bien d'autres mais les résultats recherchés sont basés sur l'aspect de diversité.\\

La simulation ne comporte que deux types d'acteurs : des individus d'une population donnée et des agent pathogènes. L'espace représenté dans la simulation est une grille bidimensionnelle permettant au individus de se déplacer librement. Ce modèle simplifié permet de représenter les interactions entre les acteurs du système. Initialement, un seul individu du système est contaminé et il s'agit de déterminer l'évolution de la situation. On cherche à observer si l'agent pathogène parvient à se propager ou non parmi la population d'individu. Une multitude d'autres paramètres peuvent être modifiés pour influencer la simulation.\\

La recherche principale est d'observer et mesurer l'évolution du système en faisant varier les diversités des acteurs, en sachant que chaque acteur du système possède un indicateur représentant son code génétique.\\

La question que l'on se pose pour ce travail est de déterminer si la diversité d'une population touchée par une épidémie modifie la propagation d'un agent pathogène ou non. Il s'agit ici d'illustrer et de quantifier cet effet à l'aide d'un modèle numérique.