\chapter{Approche du problème, méthodes, outils utilisés} \label{ch:approche}

\section{Code génétique}
Le travail se concentre sur la notion de diversité. Il faut donc un moyen pour représenter la diversité d'une population et les interactions entre différents acteurs en prenant en considération leur diversité. La méthode utilisée se base sur le principe de génome. On attribue un génome, qui représente le code génétique d'un être vivant à chaque acteur du système. Par conséquent, chaque individu et chaque agent pathogène possède un génome. Ce génome se présente sous la forme d'un entier codé sur $4$ octets. Nous avons dont $32$ bits disponibles afin de représenter le génome d'un acteur.\\

La notion de diversité d'une population peut à présent se définir par de grandes différences dans ces valeurs sur $32$ bits d'un individu à un autre. En effet, il y a une grande diversité au sein d'une population si les génomes des individus sont différents les uns des autres. Nous pouvons donc générer des séquences pour les génomes en prenant en compte le facteur de diversité du modèle. Il est possible de construire ces génomes pour la population d'individu en complémentant un certain nombre de bits d'une séquence de référence. L'idée ici est de commencer par attribuer à tous les individus un génome fixe et identique. Dans cette configuration, nous avons une diversité nulle étant donné que tout le monde a le même code génétique. Dans le cas d'une diversité non nulle, nous modifions (complémentons) certains bits de la séquence de génome des individus. Ce processus se fait aléatoirement et le nombre de complémentation dépend du paramètre de diversité du modèle. Avec cette méthode nous finissons avec des génomes déviant plus ou moins d'un certain génome de référence.\\

\section{Compatibilité entre individus et agents pathogènes}
Les génomes étant défini, il reste à définir la manière dont les agents pathogènes et les individus réagissent les uns avec les autres. Le problème est l'utilisation des génomes, que ce soit ceux des agent pathogènes ou ceux des individus. Comment gérer les interactions entre individus et agents pathogènes en se basant sur des génomes de $32$ bits ? La technique utilisée est basée sur la distance de Hamming entre les séquences des génomes. Cette méthode fournit une solution interprétable en cas de contact entre deux acteurs interagissants. Cette technique permet de traduire l'interaction de deux génomes en une action comme la contamination d'un individu sain.\\

La distance de Hamming est une notion mathématique qui permet de calculer les différences entre deux séquences binaires par exemple. Cette technique consiste simplement à compter le nombre de symboles différents pour deux suites du même nombre de symboles. Il s'agit donc de parcourir une des séquence et pour chaque indice comparer avec le symbole correspondant de l'autre séquence. Pour chaque symbole différent, la distance de Hamming est incrémentée de $1$.\\

Nous pouvons donc représenter la compatibilité entre un individu et un agent pathogène par cette distance sous forme d'un entier. Il s'agit ensuite de convertir cette valeur en une probabilité. En effet, dans notre exemple il existe uniquement $33$ valeurs possible de distance de Hamming qui caractérisent un match de génomes. Si nous appliquons une fonction de seuillage sur si peu de valeurs, les résultats vont être trop tranchés. Par conséquent nous convertissons cette distance calculé entre deux génomes en probabilité. Cette probabilité peut ensuite être évaluée ce qui détermine les actions à effectuer.\\

\section{Outils}
L'outil principal de la simulation est le C++, un langage puissant et performant. L'avantage de ce langage est qu'il est très rapide, ce qui est nécessaire pour exécuter des simulations de plus grande taille. De plus le C++ permet la programmation orientée objet, utilisé ici pour modéliser les simulations ainsi que les individus et les agents pathogènes.\\

Le langage utilisé permettant les analyses et représentations graphiques est Python. Ce langage à l'avantage d'être facile à manipuler et puissant pour faire de l'analyse. C'est donc un langage idéal pour des représentations de données. De plus Python dispose de bibliothèques logicielles comme Numpy ou encore Matplotlib qui permettent de faire du calcul scientifique ainsi que de représenter des données.\\

Le travail complet est disponible sur un dépositoire GitHub et est accessible librement. Ceci fournit de la transparence ainsi qu'un moyen de distribuer le projet. De plus toutes les versions antérieurs de la recherche y figurent.
