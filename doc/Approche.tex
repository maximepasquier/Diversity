\chapter{Approche du problème, méthodes, outils utilisés} \label{ch:approche}

\section{Approche du problème}
Il existe un grand nombre de modèles numérique simulant la propagation de pandémies. L'approche initiale était plutôt orientée vers la notion de diversité génétique. Nous voulions un moyen d'inclure et d'interpréter la diversité dans le modèle.

\section{Méthodes}
Le mécanisme de diversité se base sur l'attribution d'un code à chaque acteur du système. Ce code influence les comportements du système ainsi que l'émergence de pandémies. Le modèle numérique intègre un mécanisme simple pour interpréter ces codes.

\section{Outils}
L'outil principal de la simulation est le C++, un langage puissant et performant. L'avantage de ce langage est qu'il est très rapide, ce qui est nécessaire pour exécuter des simulations de plus grande taille. De plus le C++ permet la programmation orientée objet, utilisé ici pour modéliser les simulations ainsi que les individus et les agents pathogènes.\\

Le langage utilisé permettant les analyses et les représentations graphiques est le langage Python. Ce langage à l'avantage d'être facile à manipuler et puissant pour faire de l'analyse. C'est donc un langage idéal pour des représentations de données. De plus Python dispose de bibliothèques logicielles comme Numpy ou encore Matplotlib qui permettent de faire du calcul scientifique ainsi que de représenter des données.\\

Le travail complet est disponible sur un dépositoire GitHub et est accessible librement. Ceci fournit de la transparence ainsi qu'un moyen de distribuer le projet. De plus toutes les versions de la recherche y figurent.
