%%% 1 Introduction/ %%%
\chapter{Introduction} \label{ch:intro}

%%%%%%%%%%%%%%%%%%%%%%%%%%%%%%%%%%%%%%%%%%%%%%%%%%%%%%%%%%%%%%%

%%%%%%%%%%%%%%%%%%%%%%%%%%%%%%%%%%%%%%%%%%%%%%%%%%%%%%%%%%%%%%%

\section{first section}
Introduction to the topic, development of the rationale,
description of the thesis’s objectives and the unfolding
structure of the chapters.\\

State the scientific problem, question, goal or hypothesis.
Outline the importance, context and relevance

\section{Some examples}

\subsection{Figures}

\vspace*{-1em}
\begin{figure}[H] % the [H] here places the table right where it is in the code. If you remove it, LaTeX will decide where it fits best in terms of available space, but that might be on the next page.
\centering
\captionsetup{justification=centering}
\includegraphics[width=\textwidth]{Images/mass_conservation}
\caption[Conservation of mass]{Conservation of mass \parencite{McGinty2012}.}
\label{fig:continuity}
\vspace{-0.5em}
\end{figure}

\subsection{Tables}

\begin{table}[H]
\centering
\caption[Grid independence check dimensions]{Dimensions of grids used to test for grid independence. \label{tab:grid}}
\begin{tabular}{@{\extracolsep{\fill} } c c c c}
\toprule
Total cells & x spacing [\si{cm}] & y spacing [\si{cm}] & min/max z spacing [\si{cm}]\\ 
\midrule
760500 & 2.0 & 2.0 & 2/20\\
202800 & 4.1 & 3.8 & 2/20\\
84500 & 6.2 & 6.0 & 2/20\\
\bottomrule
\end{tabular}
\end{table}

\subsection{Equations}

\begin{equation} \label{eq:continuity}
 \frac{\partial \rho}{\partial t}
 + \frac{\partial \rho u}{\partial t}
 + \frac{\partial \rho v}{\partial t}
 + \frac{\partial \rho w}{\partial t}
 = 0
\end{equation}

To reference the equation, use this command: \ref{eq:continuity}.

\subsection{Acronyms}

To reference acronyms from Acronyms.tex, use the command \ac{AEP}.
