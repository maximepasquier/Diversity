\chapter{Conclusion} \label{ch:conclusion}

L'objectif initial du travail était de quantifier l'impacte de la diverstié d'une population sur la propagation de pandémies. Pour pourvoir répondre à cette question il a été nécessaire de couvrir d'autre points importants. Premièrement nous devions valider le modèle imlémenté en le comparant aux modèles compartimentaux déjà existant et prouvés. En plus de valider notre modèle sur des simples, nous avons pu quantifier l'impacte de divers paramètres du modèle. Les modèles compartimentaux n'ont pas de composantes spatiales, contrairement au modèle implémenté. Mais il nous a été possible de paramétrer le modèle afin de se calibrer sur les modèles compartiementaux et donc de simuler les mêmes comportement. Ensuite nous avons ajouté la notion d'espace et avons constaté et mesuré les résultats. Sur la base de ces modèles mathématiques nous avons pu explorer, comprendre et quantifier les comportements du modèle implémenté.\\

Dans un second temps nous nous sommes concentrés sur l'objectif du travail. Il s'agissait de quantifier l'impacte de la diversité. L'originalité du travail est l'utilisation de génomes simplifiés pour les acteurs du système. Ces séquences de code génétique permettent de gérer les interactions entre les acteurs. Un total de trois expériences ont été effectuées pour mesurer l'impacte de la diversité et chacune d'entre elles montre des améliorations dû à la diversité. Les améliorations observées sont de deux types différentes. Premièrement, la diversité a un impacte sur l'immunisation des individus et deuxièmement la diversité influence la réussite et la taille des pandémies.\\

Finalement nous avons mesuré l'impacte des mutations des agents pathogènes. Les mutations sont un mécanisme permettant aux agents pathogènes de devenir plus virulent. Nos résultats montrent qu'un taux de mutation non null accélère la propagation de pandémies. Par conséquent, dans des configurations identiques, les simulations avec un plus grand taux de mutation des agents pathogènes génère une pandémies de propagation plus rapide.\\

Pour conclure, le travail explore certains paramètres du modèle et étudie les comportements. Chaque phénomène observé est analysé et expliqué et permet de comprendre le fonctionnement du modèle ainsi que les mécaniques de la méthode des génomes. 