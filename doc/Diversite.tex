\chapter{Diversité} \label{ch:DIV}

\section{Mesures et méthodologie}

L'objectif du travail est de quantifier l'impacte de la diversité sur la propagation de pandémies. Cette section est dédié à la prises de mesures avec des niveaux de diversité différents et d'en constater les résultats. Afin de ne pas trop complexifier le modèle au delà des simulations SIR, nous ne modifions que le paramètre de diversité ainsi que la charge virale. D'autres paramètres du modèle pourraient être étudié mais ce chapitre ne les explore pas.\\

Toutes les musures de diversité ont été effectuées sur des systèmes aux paramètres semblables. La taille des systèmes est définie à $1264\times 1264$ avec une population de $10^5$ individus, il s'agit de systèmes en densité $\frac{1}{16}$. Un exemple de fichier de configuration est donné ci-dessous.

\begin{minted}
	[
	frame=lines,
	framesep=2mm,
	baselinestretch=1.2,
	bgcolor=LightGray,
	fontsize=\footnotesize,
	linenos
	]
	{c++}
	
	TAILLE_SYSTEME = 1264
	NOMBRE_INDIVIDUS = 100000
	ITERATIONS = 5000
	RERUN_LIMIT = 100
	FAIL_SEUIL = 30
	GENOME_INIT_I = 0
	GENOME_DIVERSITY_I = 8
	GENOME_INIT_AP = 0
	VITESSE_MUTATIONS_AP = 0
	CHARGE_VIRALE = 1
	PARAMETRE_FONCTION = 4
	CELLULE_AP = 0
	SURVIE_AP = 0
	NOMBRE_MOUVEMENT = 1
	PERFECT_MIX = true
	TEMPS_AVANT_IMMUNITE = 1
	IMMUNITE_MECANISME = true
	RESISTANCE_MECANISME = false
\end{minted}

L'exemple de fichier de configuration produit une simulation au mélange parfait avec une diversité de $8$. La configuration est similaire à celle d'une simulation SIR, la seule différence ici est l'ajout d'un paramètre de diversité non égal à $0$.\\

Toutes les simulations qui suivent se basent sur ce fichier de configuration mais avec quelques modifications.\\

Premièrement, le niveau de diversité varie d'une simulation à une autre. Les valeurs de diversité choisie sont : $4,8,16,32$. Pour rappel, une diversité de $x$ signifie que $x$ bits des génomes des individus seront complémentés et ceci alétoirement. Sans diversité les génomes des individus sont tous identiques.\\

Deuxièmement, le mode de mouvement ainsi que le nombre de mouvements varie. Seule la première simulation servant d'exemple utilise le mode de mouvement au mélange parfait, toutes les autres ont un nombre de mouvements défini. Les mouvements étudiés sont : $1,10,50$.\\

Finalement nous faisons varier le paramètre de charge virale (ici défini à 1). L'idée principale est ralentir la propagation de pandémies en diminuant la contagion des agents pathogènes. Les simulations sont effectuée sur $4$ niveaux de charge virale : $0.25,0.5,0.75,1$\\

Les résultats du chapitre se découpent en $5$ parties. La première sert d'exemple et permet d'expliquer les mécaniques derrière la diversité. Les $4$ autres parties explorent les $4$ niveaux de mouvements, les $4$ niveaux de diversité et les $4$ niveaux de charge virale.

\section{Résultats}

Chaque figure est structuré de la même manière. Il y a 3 subplots pour les 3 compartiements du modèle SIR. Les quatres courbes par compartiment représentent le nombre d'infectés de simulations aux niveaux de diversité différents.

\subsection{Mélange parfait}

Le document ne contient qu'une seule simulation à diversité non nulle et au mélange parfait. La raison pour laquelle cette voie n'est que peu explorée est parceque le mélange parfait est trop efficace pour la propagation de pandémies. Par conséquent les propagations sont beaucoup trop rapides comparé au ralentissement qu'offre la diversité. 

\begin{figure}[h]
	\centering
	\captionsetup{justification=centering}
	\includegraphics[width=.7\textwidth]{Images/SIR_diversite_mix.pdf}
	\caption{Comparaison 50 mouvements}
\end{figure}

Les résultats des simulations au mélange parfait sont très similaire aux simulations SIR et ceci est dû au fait que l'impacte de la diversité est trop faible sur cette simulation. Pour pouvoir étudier l'impact de la diversité il est nécessaire de construire des systèmes plus progressifs dans la propagation des agents pathogènes.\\

Deux choses sont à noter à propos de cette figure. La première est que la propagation est beaucoup trop rapide comme nous pouvons le voir sur le compartiement $Susceptible$. Tous les individus sont déjà passés dans le compartiment $Infected$ très tôt dans la simulation. Par conséquent dans ces configurations la propagation de la pandémie est maximale car elle atteint rapidement $100\%$ de la population.\\

Le deuxième élément se produit dans une deuxième phase et décrit l'immunisation des individus. Comme nous l'avons déjà dit, tout les individus finissent contaminés très vite. A partir de là, les individus s'immunisent mais leur immunisation dépend de la compatibilité qu'ils ont avec leur agent pathogène. Le compartiment $Recovered$ montre la différence de vitesse des acquisitions d'immunisation parmi la population et il est assez clair qu'un plus grand niveau de diversité inplique une immunisation générale plus rapide. Par conséquent la diversité permet une meilleure immunisation de la population.

\newpage

\subsection{Mouvements variable}

\begin{figure}[h]
	\centering
	\captionsetup{justification=centering}
	\includegraphics[width=.4\textwidth]{Images/SIR_diversite_1_50.pdf}
	\includegraphics[width=.4\textwidth]{Images/SIR_diversite_075_50.pdf}
	\includegraphics[width=.4\textwidth]{Images/SIR_diversite_05_50.pdf}
	\includegraphics[width=.4\textwidth]{Images/SIR_diversite_025_50.pdf}
	\caption{Comparaison 50 mouvements}
\end{figure}

\newpage

\begin{figure}[h]
	\centering
	\captionsetup{justification=centering}
	\includegraphics[width=.4\textwidth]{Images/SIR_diversite_1_10.pdf}
	\includegraphics[width=.4\textwidth]{Images/SIR_diversite_075_10.pdf}
	\includegraphics[width=.4\textwidth]{Images/SIR_diversite_05_10.pdf}
	\includegraphics[width=.4\textwidth]{Images/SIR_diversite_025_10.pdf}
	\caption{Comparaison 10 mouvements}
\end{figure}

\newpage

\begin{figure}[h]
	\centering
	\captionsetup{justification=centering}
	\includegraphics[width=.4\textwidth]{Images/SIR_diversite_1_1.pdf}
	\includegraphics[width=.4\textwidth]{Images/SIR_diversite_075_1.pdf}
	\includegraphics[width=.4\textwidth]{Images/SIR_diversite_05_1.pdf}
	\includegraphics[width=.4\textwidth]{Images/SIR_diversite_025_1.pdf}
	\caption{Comparaison 1 mouvements}
\end{figure}

\section{Analyses}

